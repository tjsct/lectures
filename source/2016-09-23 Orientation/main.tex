\documentclass{article}
\usepackage[utf8]{inputenc}
\usepackage[margin=0.8in]{geometry}
\usepackage{hyperref}

\parindent 0in
\setlength{\skip\footins}{0.5cm}


\title{Welcome to Senior Computer Team!}
\author{SCT Officers}
\date{September 23, 2016}

\begin{document}

\maketitle

\section{Introduction}

Welcome to the TJHSST Senior Computer Team, widely regarded as one of the most awesome high school computer teams in the United States! We meet weekly to practice and enhance problem-solving skills involving programming. Our focus is on the USA Computing Olympiad (USACO), but we also compete in several other programming contests. Over the year, we hope that you will all remain with us as we learn
and compete together.\footnote{This paragraph is definitely not ripped from 2013.}

\begin{itemize}
	\item Sponsor: Mr. Rudwick
	\item Captain: Kevin Geng
	\item Co-Captains: Lawrence Wang, Charles Zhao
	\item Statistician: Justin Zhang
\end{itemize}

We or any other returning members would be more than happy to answer any questions you may have about SCT, computer science, school, or life in general.


\section{First day signups}

First things first: please make sure you're signed up for both of the following.

\begin{enumerate}
    \item \textit{Mailing list.} We will use this to send out weekly announcements and lectures, so please read these emails! They are short and don't bite, we promise.
    \item \textit{Codeforces group.} This is where we will hold SCT contests. Codeforces also holds regular contests of its own, which we encourage you to do! If you do not have a Codeforces account, you should create one!
\end{enumerate}

If you aren't signed up for either of those, please use the following link to sign up for today only.

\begin{center}
\Large{\url{https://goo.gl/m5ajcq}}
\end{center}


\section{Schedule}

SCT meets every Friday during B block. (FCT and ICT meet during A block, which you should consider joining as well.) A typical block consists of lectures focused on USACO topics. As has been done in past years, we will split lectures into two groups: one for bronze/silver and one for gold/platinum. We will also occasionally provide time for coding practice.


\section{Contests}

\subsection{In-house contests}

We will regularly hold contests in our Codeforces group. Most will be practice contests, which are for your benefit. These will generally run from Friday to Wednesday, in order to give you time to think through difficult problems and discuss them with others. However, we may also hold more formal contests to help determine rankings for travel contests. We may sometimes hold contests after school as well!

\subsection{USACO}

USACO is a competition administered online with monthly contests in December, January, and February, culminating in the US Open in April. There are four divisions in USACO: bronze, silver, gold, and platinum, each with their own set of problems for each month. All contestants start in the bronze division and advance by performing well in contests; top scorers are selected for USACO training camp. Our curriculum is centered around USACO, so we strongly encourage you to participate in it.


\subsection{Travel contests}

We will select four people to represent SCT at each of the following three contests, because team registration limits prevent us from sending more.\footnote{Selections will be determined through a transparent and objective ranking process whose details are to be determined. We may take into account attendance, performance on in-house contests, or other factors.}

Each of the contests takes place on a Saturday in the spring (March or April), and is located on the corresponding university campus. They all have around 9-10 problems, and use ACM-style scoring; that is to say, the winner is determined by the total number of problems solved, and ties are broken by penalty time.

\begin{enumerate}
    \item \textit{VCU HSPC} --
    Both SCT and ICT attend. The contest length is 3 hours, and only Java is allowed.
    Website: \url{http://computer-science.egr.vcu.edu/events/highschoolcontest/}

    \item \textit{UMD HSPC} --
    Only SCT attends this contest. The contest length is 3 hours, and only Java is allowed.
    Website:
    \url{http://www.cs.umd.edu/Outreach/hsContest.shtml}
    
    \item \textit{UVA HSPC} --
    FCT, ICT, and SCT all attend. The contest length is 4 hours, and Java, C, or C++ are permitted.
    Website: \url{http://acm.cs.virginia.edu/hspc.php}
\end{enumerate}


\subsection{Other university contests}

The following contests, however, do not have a team registration limit; you may therefore form teams on your own. The number of problems and scoring system is similar.

\begin{enumerate}
    \item \textit{PClassic} -- This contest takes place every fall (November) and spring (April) at the University of Pennsylvania. You may form teams of up to four, in either the novice or standard divisions. The contest length is 4 hours, and the programming language is Java or Python. Website: \url{http://pclassic.org/}.
    \item \textit{VT HSPC} -- This contest takes place in early December, on TJ campus through the Internet. You may form teams of up to \textbf{three}. The contest length is 4 hours, and the programming language is Java, Python, C, or C++. Website: \url{https://icpc.cs.vt.edu/}
\end{enumerate}


\section{TJIOI}

This year, we're planning to revive TJIOI, a programming contest for students that was last run three years ago. More information will be forthcoming as we figure out planning details; we'll need your help!


\section{First day contest}

For the first day, we've found some practice problems for you to do on Codeforces. The practice contest will be open through next Wednesday. You can find it by navigating to the Codeforces group, or to this URL:

\begin{center}
\Large{\url{http://codeforces.com/group/M4wsRWBHyZ/contest/208940}}
\end{center}


\section{Resources}

As mentioned earlier, make sure you're signed up for our mailing list and Codeforces group! If you are not in the TJHSST SCT Facebook group, please join that, too. We post announcements there frequently.

\begin{itemize}
    \item Mailing list: \url{https://lists.tjhsst.edu/listinfo/scteam/}
    \item Codeforces group: \url{http://codeforces.com/group/M4wsRWBHyZ/}
    \item Facebook group: \url{https://www.facebook.com/groups/tjsct/}
\end{itemize}

On our website, you can find links to many excellent resources, including the above links, past lectures (including this one), and other resources, including Samuel Hsiang's excellent Crash Course Coding Companion.

\begin{center}
\Large{\url{https://activities.tjhsst.edu/sct/}}
\end{center}

If you have questions, concerns, feedback, anything — please don't hesitate to talk to any of the officers, or shoot us an email at \url{tjhsstsctcaptains@gmail.com}. We look forward to seeing you throughout the coming year!

\end{document}