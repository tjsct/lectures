\documentclass[11pt]{article}
\usepackage{amsmath}
\usepackage{amssymb}
\usepackage{amsthm}
\usepackage{array}
\usepackage{color}
\usepackage{hyperref}
\usepackage{xy}
\usepackage[margin=0.9in]{geometry}
\usepackage{verbatim}
\parindent 0in
\usepackage{tikz}

\hypersetup{
    colorlinks=true,
    urlcolor=cyan,
    linkcolor=black
}

\title{Welcome to Senior Computer Team!}
\author{SCT Officers}
\date{August 2020}

\begin{document}

\maketitle

\section{Introduction}
    Welcome to Senior Computer Team! We meet every Friday A block to learn algorithms and data structures and improve our problem solving skills via programming. Our focus is on the USA Computing Olympiad (USACO), but we also attend many other competitions. We look forward to having a great, fun year as a team! 

    
\begin{itemize}
  \item Sponsor: Mr. Eckel
  \item Captain: Justin Choi
  \item Co-Captains: Stephen Huan, Spandan Das
  \item Statistician: Sarah Zhang
\end{itemize}

\section{Schedule}
    Since we are starting school online this year, SCT will look a little bit different this year. We will be meeting Friday A Blocks (consider joining ICT, which meets B block) on Google Meets. Each week will alternate between an In-House contest for practice, and lectures to learn new concepts and applications. For a more in-depth schedule, visit our website at \url{https://activities.tjhsst.edu/sct}.
    
\section{Contests}
\subsection{In-House Contests}
       In-House Contests will be one of the most important ways to learn in SCT. We plan to have many In-House Contests this year in multiple formats to show students weaknesses and strengths of their algorithmic programming knowledge. There will two types of IHC: A live blitz round, which will be held during 8th period blocks and focused on speed, and a longer comprehensive round, which would be open all weekend long and focus on harder problems. We will alternate between these two formats this year to give practice with speed and problem solving.
\subsection{USACO}
    USACO, the most popular and prestigious computing contest in the US, is administered online through four contests in December, January, February, and ending with an US Open in March. There are four divisions: Bronze, Silver, Gold, and Platinum. Everyone starts out in the Bronze Division and top competitors are selected for USACO camp. We highly recommend everyone to do this as much of our material overlaps with it, and one can get a lot out of the contests.  

\section{Changes}
    This year, we are implementing a few changes compared to last year's SCT club. 
    
\begin{itemize}

    \item All contests will be online this year so we will try to let you know about as many as possible so that you can register with your own teams. For some contests, we will make teams based on rankings from In-House Contests.
    \item We will hold In-House contests with ICT every two weeks, alternating between a short blitz format, and a longer weekend format. Lectures will be biweekly and will first introduce a topic and then go deep into applications
    \item We are introducing a ranking system that would be used for online competitions and maybe officer positions. These rankings would include In-House Contest Performance and USACO Division.
    \item We are still determining whether guest lectures shall be a requirement or not to run for an officer position. If you want to lecture, email the officers, and we can work something out.
\end{itemize}


\section{Resources}
\subsection{Contact}
    Make sure to sign up for our mailing list and join our Facebook group! We will send out weekly announcements and updates. If you have any questions, feel free to contact us.
\begin{itemize}
  \item Website: \url{https://activities.tjhsst.edu/sct}
  \item Email: \href{mailto:tjhsstsctcaptains@gmail.com}{tjhsstsctcaptains@gmail.com}
  \item Facebook Group: \url{https://www.facebook.com/groups/tjsct}
\end{itemize}

\subsection{Problem Archives}
    Solving problems is \textbf{the most important} part to learning new concepts. We will try to showcase a few problems relevant to each lecture topic at the end of lectures, but it is important that you do more problems!
\begin{itemize}
  \item USACO Training Pages
  \item USACO Past Contests Archives (Effectively learn through explanations)
  \item Codeforces
  \item Online Judges (UVA, Kattis, CodeChef)
\end{itemize}

\subsection{Competitive Programming Resources}
\begin{itemize}
  \item usaco.org
  \item Wikipedia
  \item geeksforgeeks.org
  \item \textit{Competitive Programming} by Steve and Felix Hamlin 
  \item \textit{Algorithms} by Sedgewick and Wayne
\end{itemize}
\end{document}
