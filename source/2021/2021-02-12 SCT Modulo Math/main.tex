\documentclass{article}
\usepackage[utf8]{inputenc}
\usepackage{hyperref}
\hypersetup{
    colorlinks=true,
    linkcolor=blue,
    filecolor=magenta,      
    urlcolor=blue,
}

\title{SCT Modular Math}
\author{Andrew Wang}
\date{February 2021}

\begin{document}

\maketitle

\section{Introduction}
    \hspace{1em} \quad Modular Arithmetic works with remainders instead of integers. For example,
    $$
    9 \bmod 5 = 4
    $$
    since the remainder of $9$ divided by $5$ is $4$.
    In contests, you'll often see modular arithmetic used to avoid dealing with large numbers that overflow. 
\subsection{Modular Arithmetic Properties}
    \hspace{1em} \quad Some common properties in modular arithmetic:
        $$
        (a+b) \bmod m = (a \bmod m + b \bmod m) \bmod m
        $$
        
        $$
        (a-b) \bmod m = (a \bmod m - b \bmod m) \bmod m
        $$
        
        $$
        (a \cdot b) \pmod{m} = ((a \bmod m) \cdot (b \bmod m)) \bmod m
        $$
        
        $$
        a^b \bmod {m} = (a \bmod m)^b \bmod m
        $$
\section{Optimizations}
    \hspace{1em} \quad Taking the modulo of a number over and over again can be very time-costly due to the high constant factor.
    \subsection{Pre-Calculation}
        \hspace{1em} \quad If we're taking the modulo of some set of numbers over and over again(i.e powers of numbers), it may be faster to pre-calculate the modulo of each of these numbers beforehand.
    \subsection{Addition and Subtraction}
        \hspace{1em} \quad Since the $\%$ operator has a much higher constant factor compared to more elementary operators like addition or subtraction, always use addition and subtraction when you have the choice. For example,
        $$
            9 \bmod 5 = 9-5.
        $$
\section{Tips \& Tricks}
    \begin{itemize}
        \item Take the modulo of each number before performing operations to prevent overflow.
        \item While debugging, if you come across a negative number it almost always means overflow.
    \end{itemize}
\section{Euler's Totient Theorem}
\hspace{1em} \quad Euler's Totient Function is commonly seen in number theory. It states:
    $$
    a^{ \varphi (n)} = 1 \bmod n.
    $$
\subsection{Fermat's Little Theorem}
    \hspace{1em} \quad Fermat's Little Theorem is an extensions of Euler's Totient where $n$ is a prime number. This simplifies to:
    $$
        a^{p-1} = 1 \bmod p
    $$
    where $p$ is any prime.
    \subsection{Modular Inverse}
        \hspace{1em} \quad Dividing can be very difficult while in some mod $n$. For example,
        $$
        (9/3) \bmod 5 \neq ((9 \bmod 5)/(3 \bmod 5)) \bmod 5
        $$
        Luckily, using modular inverses, we can safely divide numbers without worrying about mistakes during division. The modular inverse of a number is equivalent to the reciprocal but in a certain mod. 
        $$
        a/b \bmod m = a*i \bmod m
        $$
        where i is the modular inverse of b. The modular inverse of $b \bmod m$, is equivalent to $b^{m-1} \bmod m$ since by Fermat's:
        $$
        b^{m-1} \bmod m \equiv 1 \bmod m
        $$
        if m is prime. Finding $b^{m-1}$ is a much simpler task which can be solved using \href{https://cp-algorithms.com/algebra/binary-exp.html}{binary exponentiation} or pre-calculating the powers of a number as mentioned before.
    \subsection{Example Problem: \href{https://cses.fi/problemset/task/1079}{Binomial Coefficients}}
        \hspace{1em} \quad
        \begin{enumerate}
            \item Pre-calculate factorials in an array
            \item $\frac{a!}{b!(a-b)!} \bmod m = a! \cdot inverse(b!) \cdot inverse((a-b)!) \bmod m$
            \item calculate the inverses using binary exponentation
        \end{enumerate}
\section{Miscellaneous}
    Sometimes, problem-setters mess up!
    \subsection{Challenge Problem: \href{http://www.usaco.org/index.php?page=viewproblem2&cpid=946}{USACO Gold walk}}
    \hspace{1em} \quad Although the intended solution is an $O(N^2)$ MST, this can be easily fake-solved in $O(1)$ using some modular arithmetic. The general idea is to maximize the function:
        $$
        \min\limits_{x,y \, \textrm{in different groups}} (2019201997 - 84 x - 48 y).
        $$
    This can be done by placing the smallest $k-1$ in their own distinct group and the other larger numbers together in one group. Simplifying gives us the equation:
    $$
    2019201997 - 84 (k-1) - 48 n
    $$
    (For a full proof check out the \href{https://usaco.guide/solutions/usaco-946}{full solution})
\section{Resources}
    \subsection{References}
    \begin{itemize}
        \item \href{https://usaco.guide/gold/modular}{USACO Guide Modular Arithmetic}
        \item \href{https://usaco.guide/CPH.pdf#page=211}{CPH Modular Math}
    \end{itemize}
        
    \subsection{Problens}
        \begin{itemize}
            \item \href{https://usaco.guide/gold/modular}{USACO Guide Problem set} (At the end of the page)
        \end{itemize}

\end{document}
