\documentclass{article}
\usepackage[utf8]{inputenc}

\usepackage{listings}
\usepackage[hyperfootnotes=false]{hyperref}
\usepackage{exercise}

\usepackage{verbatim}

\usepackage{algorithm}
\usepackage{algorithmicx}
\usepackage[noend]{algpseudocode}

\usepackage[toc,page]{appendix}

\usepackage[secthm,mdthm]{evan}

\lhead{SCT Officers}

\usepackage{graphicx}

\usepackage{color}

\begin{comment} % for black and white printing; see below
\definecolor{myseagreen}{RGB}{88,197,191}
\definecolor{mysalmon}{RGB}{255,160,122}
\definecolor{myred}{RGB}{255,102,102}
\definecolor{mypurple}{RGB}{225,145,255}
\end{comment}
\definecolor{myblack}{RGB}{0,0,0}
\definecolor{mywhite}{RGB}{255,255,255}

% for black and white printing
\definecolor{myseagreen}{rgb}{1.0,1.0,1.0}
\definecolor{mysalmon}{rgb}{1.0,1.0,1.0}
\definecolor{myred}{rgb}{1.0,1.0,1.0}
\definecolor{mypurple}{rgb}{1.0,1.0,1.0}

\usepackage{tikz}

\usetikzlibrary{calc,shapes.multipart,chains,arrows,positioning}

\tikzstyle{vertex}=[draw,fill=myseagreen,circle,minimum size=24pt,inner sep=0pt]

\tikzstyle{splitvertex}=[draw,fill=myseagreen,circle split,minimum size=24pt]

\usetikzlibrary{
  shapes.multipart,
  matrix,
  positioning,
  shapes.callouts,
  shapes.arrows,
  shapes.geometric,
  decorations.shapes,
  shapes,
  fit,
  arrows,
  positioning,
  trees,
  mindmap,
  calc}

\tikzset{
    squarecross/.style={
        draw, rectangle,minimum size=18pt, fill=myseagreen,
        inner sep=0pt, text=black,
        path picture = {
            \draw[black]
            (path picture bounding box.north west) -- 
            (path picture bounding box.south east)
            (path picture bounding box.south west) -- 
            (path picture bounding box.north east);
        }
    }
}

\definecolor{codegreen}{rgb}{0,0.6,0}
\definecolor{codegray}{rgb}{0.5,0.5,0.5}
\definecolor{codepurple}{rgb}{0.58,0,0.82}
% \definecolor{backcolour}{rgb}{0.95,0.95,0.92}
\definecolor{backcolour}{rgb}{1,1,1}

\lstnewenvironment{mylstlisting}[1][]%
  {\noindent\minipage{\linewidth}\medskip 
   \lstset{
    backgroundcolor=\color{backcolour},   
    commentstyle=\color{codegreen},
    keywordstyle=\color{magenta},
    numberstyle=\tiny\color{codegray},
    stringstyle=\color{codepurple},
    basicstyle=\footnotesize,
    breakatwhitespace=false,         
    breaklines=true,                 
    captionpos=t,                    
    keepspaces=true,                 
    numbers=left,                    
    numbersep=5pt,                  
    showspaces=false,                
    showstringspaces=false,
    showtabs=false,                  
    tabsize=4,
   basicstyle=\ttfamily\footnotesize,
   frame=single,#1}}
  {\endminipage}

\usepackage{footnote}
\makesavenoteenv{tabular}
\makesavenoteenv{table}

\renewcommand{\arraystretch}{1.5}

\title{Sample I/O}
\author{SCT Officers}
\date{October 2015}

\begin{document}

\maketitle

The following are some I/O examples in various languages to get you started.

All programs read in a number $N$ from the first line, proceed to read in $N$ more numbers from the second line, and print their sum.

Sample input:

\begin{verbatim}
4
1 3 5 8
\end{verbatim}

Sample output:

\begin{verbatim}
17
\end{verbatim}

\section{Java}

\lstset{language=Java}

Here I use a \texttt{BufferedReader} instead of a \texttt{Scanner}. It's much faster and much more reliable in the contest environment.

\subsection{Standard I/O}

\begin{mylstlisting}
import java.util.*;
import java.io.*;

class sum {
    public static void main(String[] args) throws IOException {
        BufferedReader f = new BufferedReader(new InputStreamReader(System.in));
	    int N = Integer.parseInt(f.readLine()); // read whole line
	    int[] num = new int[N];
	    StringTokenizer st = new StringTokenizer(f.readLine()); // split line by white space
    	for(int k = 0; k < N; ++k) {
    		num[k] = Integer.parseInt(st.nextToken());
    	}
    	int sum = 0;
    	for(int k = 0; k < N; ++k) {
    		sum += num[k];
    	}
    	System.out.println(sum);
    	System.exit(0);
    }
}
\end{mylstlisting}

\subsection{File I/O}

\begin{mylstlisting}
import java.util.*;
import java.io.*;

class sum {
    public static void main(String[] args) throws IOException {
        BufferedReader f = new BufferedReader(new FileReader("sum.in"));
		PrintWriter out = new PrintWriter(new BufferedWriter(new FileWriter("sum.out")));
	    int N = Integer.parseInt(f.readLine()); // read whole line
	    int[] num = new int[N];
	    StringTokenizer st = new StringTokenizer(f.readLine()); // split line by white space
    	for(int k = 0; k < N; ++k) {
    		num[k] = Integer.parseInt(st.nextToken());
    	}
    	int sum = 0;
    	for(int k = 0; k < N; ++k) {
    		sum += num[k];
    	}
    	out.println(sum);
    	out.close(); // don't forget this!
    	System.exit(0);
    }
}
\end{mylstlisting}

\section{C++}

\lstset{language=C++}

Here I will use C++-style I/O. If you prefer C-style I/O, that's fine too.

\subsection{Standard I/O}

The first two lines here are to speed up input. They are considered bad coding practice outside of the contest environment but are essential to get your times down if I/O is large. Note, however, that if you unlink with C-style I/O, you may not use \texttt{scanf()} and \texttt{printf()}, etc. Bad things will happen.

\begin{mylstlisting}
#include <iostream>
#include <fstream>

int num[100005];

int main() {
    std::ios_base::sync_with_stdio(0); // unlink C-style I/O
    std::cin.tie(0); // unlink std::cout
	std::cin >> N;
	for(int k = 0; k < N; ++k) {
		std::cin >> num[k];
	}
	int sum = 0;
	for(int k = 0; k < N; ++k) {
		sum += num[k];
	}
	cout << sum << "\n";
	return 0;
}
\end{mylstlisting}

\subsection{File I/O}

\begin{mylstlisting}
#include <iostream>
#include <fstream>

int num[100005];

int main() {
	std::ifstream fin("palpath.in");
	std::ofstream fout("palpath.out");
	fin >> N;
	for(int k = 0; k < N; ++k) {
		fin >> num[k];
	}
	int sum = 0;
	for(int k = 0; k < N; ++k) {
		sum += num[k];
	}
	fout << sum << "\n";
	fin.close();
	fout.close(); // don't forget this!
	return 0;
}
\end{mylstlisting}

\section{Python}

\lstset{language=Python}

\subsection{Standard I/O}

\begin{mylstlisting}
n = int( input() ) # input() grabs the whole line
nums = input().strip() # removes extra spaces at beginning and end, also \n
nums = nums.split() # splits at the spaces to turn it into an array of strings
nums = [int(stng) for stng in nums] #turn them into ints
print( sum( nums ) )
\end{mylstlisting}

\subsection{File I/O}

\begin{mylstlisting}
file = open('input.txt', 'r') # r for read
out = open('output.txt', 'w') # w for write

n = int(file.readline().strip()) # strip() isn't always necessary, but it's a good habit
nums = file.readline().strip()
nums = nums.split() # splits at the spaces to turn it into an array of strings
nums = [int(stng) for stng in nums] #turn them into ints
out.write( str( sum( nums ) ) + '\n' ) # str() and + are necessary because write() takes only a single string

file.close()
out.close()
\end{mylstlisting}

\end{document}
