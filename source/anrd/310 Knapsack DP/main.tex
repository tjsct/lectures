\documentclass{article}
\usepackage[utf8]{inputenc}
\usepackage{hyperref}
\usepackage[margin=0.9in]{geometry}
\usepackage{verbatim}
\parindent 0in
\usepackage{listings}
\usepackage{tikz}
\def\checkmark{\tikz\fill[scale=0.4](0,.35) -- (.25,0) -- (1,.7) -- (.25,.15) -- cycle;} 

\definecolor{dkgreen}{rgb}{0,0.6,0}
\definecolor{gray}{rgb}{0.5,0.5,0.5}
\definecolor{mauve}{rgb}{0.58,0,0.82}

\lstset{frame=tb,
  language=Java,
  aboveskip=3mm,
  belowskip=3mm,
  showstringspaces=false,
  columns=flexible,
  basicstyle={\small\ttfamily},
  numbers=none,
  numberstyle=\tiny\color{gray},
  keywordstyle=\color{blue},
  commentstyle=\color{dkgreen},
  stringstyle=\color{mauve},
  breaklines=true,
  breakatwhitespace=true,
  tabsize=3
}
\title{10-25-19 Knapsack DP}
\author{Patrick Zhang}
\date{October 2019}

\begin{document}

\maketitle
\section{Dynamic Programming Introduction}
Dynamic Programming (DP) is one of the most tested concepts at he USACO Gold and Platinum levels. About 1/3 of USACO Gold problems are DP problems. 

DP is solving a problem using previously solved subproblems. It is similar, but objectively different than Greedy.

\section{Procedure}
To solve DP problems, you should follow these general steps. 

\begin{enumerate}
    \item Determine your states.
    \item Set initial states.
    \item Transition between states.
    \item Reach final state and get answer.
\end{enumerate}

\section{Example 0: Fibonacci}
\subsection{Problem}
Given $N (1 \leq N \leq 100000),$ output the $N$th fibonacci number

\subsection{Solution}
Make an array called dp, which is size $N+1.$ dp[$K$] equals $K$th fibonacci number. When we reach a $K,$ we know dp[$M$] for all $1 \leq M < K,$ and we can use those values to calculate dp[$K$] 

Our initial states are dp[$0$] = 1 and dp[$1$] = 1.

Loop $i$ from $2$ to $N.$ To transition between states, we will set dp[$i$] to dp[$i-1$] + dp[$i-2$]. 

If $N = 8,$ our dp array will end up looking like this: 
\begin{table}[h]
\begin{tabular}{llllllllll}
                        & 0                      & 1                      & 2                      & 3                      & 4                      & 5                      & 6                       & 7                       & 8                       \\ \cline{2-10} 
\multicolumn{1}{l|}{dp} & \multicolumn{1}{l|}{1} & \multicolumn{1}{l|}{1} & \multicolumn{1}{l|}{2} & \multicolumn{1}{l|}{3} & \multicolumn{1}{l|}{5} & \multicolumn{1}{l|}{8} & \multicolumn{1}{l|}{13} & \multicolumn{1}{l|}{21} & \multicolumn{1}{l|}{34} \\ \cline{2-10} 
\end{tabular}
\end{table}

\section{Example 1: The Knapsack Problem}
\subsection{Problem}
The Knapsack Problem is one of the most fundamental DP problems. It goes as follows:

You have a bag (a ``knapsack''), and three different types of coins with the values of 1, 3, and 5. How many ways are there to put the coins in the bag such that the coins sum to $N$ ($0 \leq N \leq 10^5)?$ For this problem, order matters, meaning putting a 1 coin then a 3 coin is different than putting a 3 coin then a 1 coin.

\subsection{Solution}
Let $f(N)$ be the number of ways to add coins such that the sum is $N.$ Note that $f(N) = f(N-1) + f(N-3) + f(N-5).$ This is because you can reach a sum of $N$ by adding a 1 coin to $N-1,$ a 3 coin to $N-3$, or a 5 coin to $N-5.$ Obviously, $f(N) = 0$ if $N < 0,$ so we know that we can calculate $f(N)$ if we know $f(M)$ for all $0 \leq M < N.$ 

\subsection{Walkthrough}
Let $N$ be $9.$

We need to set an initial state. For this problem, the initial state is $f(0) = 1.$ This means that there is $1$ way to add coins that sum to $0$ (which is to not add any coins). $f(1) = f(0) + f(-2) + f(-3),$ which equals $1.$

We can continue this way until we reach $9.$

\begin{center}
    \begin{tabular}{c|c}
        \hline
        $N$& $f(N)$ \\
        \hline
        $0$ & $1$ \\
        \hline
        $1$ & $1$ \\
        \hline
        $2$ & $1$ \\
        \hline
        $3$ & $2$ \\
        \hline
        $4$ & $3$ \\
        \hline
        $5$ & $5$ \\
        \hline
        $6$ & $8$ \\
        \hline
        $7$ & $12$ \\
        \hline
        $8$ & $19$ \\
        \hline
        $9$ & $30$ \\
    \end{tabular}
\end{center}

We can test our solution on smaller values to ensure that our solution is correct. For instance, $f(5) = 5,$ which we know is true because there are $5$ ways to add coins such that they sum to $5: (1,1,1,1,1), (1,1,3), (1,3,1), (3,1,1), (5)$

Our answer is thus $f(9) = \fbox{30}.$

\subsection{Walkthrough 2: Forward DP}
That solution is called "Backward DP". When we reached a state of $f(N),$ every $f(M)$ for $0 \leq M < N$ was solved, and we used those states to solve $f(N).$ In Forward DP, when we reach a state of $f(N), f(N)$ will already be completely solved. We will use $f(N)$ to help solve the states that succeed it.

Every time we reach a state of $f(N),$ we will run the following operations:

$$f(N+1) += f(N)$$
$$f(N+3) += f(N)$$
$$f(N+5) += f(N)$$

(Don't forget to check for ArrayOutOfBoundsException!)

\section{Example 2: The Knapsack Problem 2}
\subsection{Problem}
This problem is the same as the previous one, only order doesn't matter, so adding a 1 coin then a 3 coin is the same as adding a 3 coin then a 1 coin. One way to ensure you avoid duplicates is to only add coins greater than or equal to in value to the coin in the bag with the maximum value. For example, if you've already added a 3 coin, you can only add a 3 coin or a 5 coin. You may have realized that we will need to somehow store the maximum coin that we have already added. In order to do this, we will need more complicated states.

\subsection{Solution}
For the first problem, we have an array dp of size [$N$], where dp[$K$] represented the state where the sum of the coins is $K$. In this problem, we will have a two dimensional matrix dp of size [$N$][$3$], where the state of dp[$K$][$J$] represents the state where the sum of the coins is $K$ and the maximum coin used is the $J$th highest coin. 

In the previous problem, we added all of the coins to each state. In this problem, we will only add the coins bigger or equal to the $J$th coin.

The answer to the problem is dp[$N$][$0$] + dp[$N$][$1$] + dp[$N$][$2$].

\subsection{Walkthrough}
Let $N$ be $9.$

We will be using Forward DP for this walkthrough. For simplicity, we will set dp[$1$][$0$], dp[$3$][$1$], and dp[$5$][$2$] to $1$ as our initial states. They represent adding a 1 coin, 3 coin, and 5 coin, respectively. 

After two interations, we end up with the following matrix:

\begin{table}[h]
\begin{tabular}{llllllllll}
                       & 1                      & 2                      & 3                      & 4                      & 5                      & 6                      & 7                     & 8                     & 9                     \\ \cline{2-10} 
\multicolumn{1}{l|}{0} & \multicolumn{1}{l|}{1} & \multicolumn{1}{l|}{1} & \multicolumn{1}{l|}{}  & \multicolumn{1}{l|}{}  & \multicolumn{1}{l|}{}  & \multicolumn{1}{l|}{}  & \multicolumn{1}{l|}{} & \multicolumn{1}{l|}{} & \multicolumn{1}{l|}{} \\ \cline{2-10} 
\multicolumn{1}{l|}{1} & \multicolumn{1}{l|}{}  & \multicolumn{1}{l|}{}  & \multicolumn{1}{l|}{1} & \multicolumn{1}{l|}{1} & \multicolumn{1}{l|}{}  & \multicolumn{1}{l|}{}  & \multicolumn{1}{l|}{} & \multicolumn{1}{l|}{} & \multicolumn{1}{l|}{} \\ \cline{2-10} 
\multicolumn{1}{l|}{2} & \multicolumn{1}{l|}{}  & \multicolumn{1}{l|}{}  & \multicolumn{1}{l|}{}  & \multicolumn{1}{l|}{}  & \multicolumn{1}{l|}{1} & \multicolumn{1}{l|}{1} & \multicolumn{1}{l|}{} & \multicolumn{1}{l|}{} & \multicolumn{1}{l|}{} \\ \cline{2-10} 
\end{tabular}
\end{table}

After we finish iterating, we end up with the following matrix:

\begin{table}[h]
\begin{tabular}{llllllllll}
                       & 1                      & 2                      & 3                      & 4                      & 5                      & 6                      & 7                      & 8                      & 9                      \\ \cline{2-10} 
\multicolumn{1}{l|}{0} & \multicolumn{1}{l|}{1} & \multicolumn{1}{l|}{1} & \multicolumn{1}{l|}{1} & \multicolumn{1}{l|}{1} & \multicolumn{1}{l|}{1} & \multicolumn{1}{l|}{1} & \multicolumn{1}{l|}{1} & \multicolumn{1}{l|}{1} & \multicolumn{1}{l|}{1} \\ \cline{2-10} 
\multicolumn{1}{l|}{1} & \multicolumn{1}{l|}{}  & \multicolumn{1}{l|}{}  & \multicolumn{1}{l|}{1} & \multicolumn{1}{l|}{1} & \multicolumn{1}{l|}{1} & \multicolumn{1}{l|}{2} & \multicolumn{1}{l|}{2} & \multicolumn{1}{l|}{2} & \multicolumn{1}{l|}{3} \\ \cline{2-10} 
\multicolumn{1}{l|}{2} & \multicolumn{1}{l|}{}  & \multicolumn{1}{l|}{}  & \multicolumn{1}{l|}{}  & \multicolumn{1}{l|}{}  & \multicolumn{1}{l|}{1} & \multicolumn{1}{l|}{1} & \multicolumn{1}{l|}{1} & \multicolumn{1}{l|}{2} & \multicolumn{1}{l|}{2} \\ \cline{2-10} 
\end{tabular}
\end{table}

The answer is $1+3+2=\fbox{6}.$ We know the answer is correct because there are $6$ ways to reach $9: (1,1,1,1,1,1,1,1,1), (1,1,1,1,1,1,3), (1,1,1,3,3), (3,3,3),$ $(1,1,1,1,5), (1,3,5).$ Note that there is $1$ way to reach $9$ with only 1 coins, $3$ ways using 1 and 3 coins, and $2$ ways using 1, 3, and 5 coins.

\subsection{Code}
\begin{lstlisting}
#include <bits/stdc++.h>

using namespace std;

//N will be <10^5 so just make a big matrix.
int dp[100005][3];

int main(){
   ios::sync_with_stdio(false);
   cin.tie(0);
   
   /*
   Use this for file input/output (like for USACO),
   and use fin and fout instead of cin and cout
   ifstream fin (".in");
   ofstream fout (".out");
   */
   
   
   int N;
   cin >> N;
   
   //initialize putting a 1 coin, 3 coin, and 5 coin.
   dp[1][0] = 1;
   dp[3][1] = 1;
   dp[5][2] = 1;
   
   for(int k = 1; k < N; k++){
        for(int j = 0; j < 3; j++){
            if(dp[k][j] == 0) continue;
            if(j <= 0 && k+1 <= N) dp[k+1][0] += dp[k][j];
            if(j <= 1 && k+3 <= N) dp[k+3][1] += dp[k][j];
            if(j <= 2 && k+5 <= N) dp[k+5][2] += dp[k][j];
        }
   }
   
   int answer = dp[N][0] + dp[N][1] + dp[N][2];
   cout << answer << endl;
   
   
   return 0;
}
\end{lstlisting}



\section{The Best Way to Practice Dynamic Programming}
Go to \url{https://codeforces.com/problemset/page/1?tags=dp&order=BY_SOLVED_DESC.} Codeforces is a competitive programming platform, like USACO, that has a database of thousands of problems. That link filters out all DP problems and sorts them by difficulty. Start at the top and solve $1$ out of every $2$ or $3$ problems.




\end{document}
