\documentclass[11pt]{article}
\usepackage{amsmath}
\usepackage{amssymb}
\usepackage{amsthm}
\usepackage{array}
\usepackage{color}
\usepackage{hyperref}
\usepackage{xy}
\usepackage[margin=0.9in]{geometry}
\usepackage{verbatim}
\parindent 0in

\usepackage{tikz}
\def\checkmark{\tikz\fill[scale=0.4](0,.35) -- (.25,0) -- (1,.7) -- (.25,.15) -- cycle;} 

\def\returning{}
%\def\returning{\comment}

\title{Senior Computer Team Orientation}
\author{SCT Officers }
\date{September 2019}

\begin{document}

\maketitle

\section{Introduction}
    Welcome to Senior Computer Team, one of the coolest computer science clubs at TJ! We meet every Friday A block to learn algorithms and data structures and improve our problem solving skills via programming. Our focus is on the USA Computing Olympiad (USACO), but we also attend many other college competitions throughout the year. We look forward to having a great, fun year as a team! 

\begin{itemize}
  \item Sponsor: Mr. Eckel
  \item Captain: Richard Zhan
  \item Co-Captains: Patrick Zhang, Neeyanth Kopparapu
  \item Statistician: Havish Malladi
\end{itemize}

\section{Schedule}
    SCT  usually meets Friday A block (consider joining ICT, which meets B block). A typical block consists of two lectures on competitive programming topics. Students can choose which lecture to attend. Every few meetings, we will have an in-house contest, which will give members a chance to practice the material taught.
    For a more in-depth schedule, go to https://activities.tjhsst.edu/sct/schedule.php and for more general information visit https://activities.tjhsst.edu/sct

\section{Contests}
\subsection{In-house Contests}
    In-house contests will be one of the most important ways to learn in SCT. We plan to have three to four in-house contests this year on specific skills to show students weaknesses and strengths of their algorithmic programming knowledge. Each in-house contest's format will be released before each contest.
\subsection{USACO}
    USACO, the most popular and prestigious computing contest in the US, is administered online through four contests in December, January, February, and ending with an US Open in March. There are four divisions: Bronze, Silver, Gold, and Platinum. Everyone starts out in the Bronze Division and top competitors are selected for USACO camp. We highly recommend everyone to do this as much of our material overlaps with it, and one can get a lot out of the contests.       
\subsection{Travel/Additional Contests}
    We also participate in a variety of other contests, which are held in the spring. Each contest is formatted ACM-style, with the winners solving the most problems and ties broken by time. Teams of 3 to 4 are selected to participate at: UVA, VCU, and Codequest, and others. Additionally, every early December on TJ campus, VT hosts an online programming contest. You may form up to teams of three. This year, SCT is looking to expand the number of people we send to contests and the contests we attend, so be on the look out for information about  
\section{TJIOI}
    This year, TJIOI will be happening \textbf{in the fall} (tentatively November 9th) so we will be needing your help making the event a success! Look in the near future for updates regarding:
    \begin{itemize}
        \item Volunteering (You can get service hours and have fun!)
        \item Problem Writing and Organization
    \end{itemize}
    If you want to help with TJIOI, please ask/message an officer ASAP!
\section{Topics}
    The main focus of the Intermediate and Advanced lectures is to help students promote to the Gold and Platinum divisions of USACO, respectively. We plan to cover all of the commonly seen topics for those two divisions before the first contest in December. After that, we will cover more advanced, Platinum-level topics.
    
    Common USACO topics:
    
    \begin{itemize}
    \item Silver
    \begin{itemize}
        \item Breadth First Search and Depth First Search
        \item Floodfill
        \item Prefix Sums
        \item Greedy Algorithms
    \end{itemize}
    \item Gold
    \begin{itemize}
        \item Dynamic Programming
        \item Graphs (Djikstra's, Disjoint Set Union, Minimum Spanning Tree)
    \end{itemize}
    \item Platinum
    \begin{itemize}
        \item Advanced Dynamic Programming
        \item Segment Trees and Ranged Queries
        \item Math
    \end{itemize}
    \end{itemize}

\section{Changes}
    This year, we are implementing a few changes compared to last year's SCT club. 
\begin{itemize}
    \item We plan to attend more travel contests this year to give more people an opportunity to attend these contests. In the near future, we will be releasing a list of "nearby" contests (within a four-hour drive of TJ). 
    %In-house contest performance will be w0w a dash  used to select teams. We are reaching out to farther contests, and we will attend them determined by how far chaperones are willing to drive (we are currently looking at contests within four hours of TJ).
    \item Guest lectures, which are required to run for an officer position, will not be clustered at the end of the year, but instead distributed throughout second semester to engage interests in the lectures. The requirements for running has also changed, and now includes:
    \begin{itemize}
        \item 1 \textbf{solo} guest lecture
        \item 12 SCT meetings attended
    \end{itemize}
    In order to vote, you will need to attend \textbf{6} SCT meetings.
    \end{itemize}

\section{Resources}
\subsection{Contact}
    Make sure to sign up for our mailing list and join our Facebook group! We will send out weekly announcements and updates. If you have any questions, feel free to contact us.
\begin{itemize}
  \item Website: https://activities.tjhsst.edu/sct/
  \item Email: tjhsstsctcaptains@gmail.com
  \item Facebook Group: https://www.facebook.com/groups/tjsct/
\end{itemize}

\subsection{Problem Archives}
    Solving problems is \textbf{the most important} part to learning new concepts. We will try to showcase a few problems relevant to each lecture topic at the end of lectures, but it is important that you do more problems!
\begin{itemize}
  \item USACO Training Pages
  \item USACO Past Contests Archives (Effectively learn through explanations)
  \item Codeforces
  \item Online Judges (UVA, Kattis, CodeChef)
\end{itemize}

\subsection{Competitive Programming Resources}
\begin{itemize}
  \item usaco.org
  \item codeforces.com
  \item https://github.com/bqi343/USACO
  \item geeksforgeeks.org
  \item \textit{Competitive Programming} by Steve and Felix Hamlin 
  \item \textit{Algorithms} by Sedgewick and Wayne
\end{itemize}
\end{document}
