\documentclass{article}
\usepackage[utf8]{inputenc}

\title{Standard Data Structures: Sets and Maps}
\author{ICT and SCT Captains}
\date{October 2020}

\begin{document}

\maketitle

\section{Introduction}
    Most programming languages come with built-in data structures that are useful in competitive programming. Of those data structures, this lecture will go over sets and maps. 
    
\section{Sets}
    A set is a data structure that allows quick random access, addition, and removal of values, but can't contain repeats of items.
\subsection{Unordered Sets}
    Unordered Sets(HashSet in java) use hashing to allow very quick access to elements. The downside is that we don't know the order of the elements in the set.
    \newline 
    \newline
    Effciencies: 
    \begin{itemize}
        \item Adding a value: $O(1)$
        \item Removing a value: $O(1)$
        \item Checking if the set contains a value: $O(1)$
        \item Looping through the set: $O(N)$
    \end{itemize}
    
\subsection{Ordered Sets}
    Ordered Sets(TreeSet in java) use a binary search tree to allow quick access to elements. Values in the set are kept in sorted order.
    \newline 
    \newline
    Effciencies: 
    \begin{itemize}
        \item Adding a value: $O(log(N))$
        \item Removing a value: $O(log(N))$
        \item Checking if the set contains a value: $O(log(N))$
        \item Finding the smallest or largest value: $O(log(N))$
        \item Finding the smallest value bigger than a given value: $O(log(N))$
        \item Finding the largest value smaller than a given value: $O(log(N))$
        \item Looping through the set: $O(N)$
    \end{itemize}

\section{Maps}
    Maps are like set, but instead of storing values, we store keys. Each key "maps", or is linked to a value, and we must access values using their keys. Like in sets, keys can't repeat. However, the values the keys map to can.
\subsection{Unordered Maps}
    Unordered Maps(HashMaps in java) are pretty much just the map version of an unordered set.
    \newline 
    \newline
    Effciencies: 
    \begin{itemize}
        \item Adding a key-value pair: $O(1)$
        \item Removing a key-value pair: $O(1)$
        \item Updating a key-value pair: $O(1)$
        \item Checking if the map contains a key: $O(1)$
        \item Checking if the map contains a value: $O(N)$
        \item Getting the value for a certain key; $O(1)$
        \item Looping through the map: $O(N)$
    \end{itemize}
    
\subsection{Ordered Sets}
    Ordered Maps(TreeMaps in java) are like their set counterparts. Keys are kept in sorted order.
    \newline 
    \newline
    Effciencies: 
    \begin{itemize}
        \item Adding a key-value pair: $O(log(N))$
        \item Removing a key-value pair: $O(log(N))$
        \item Updating a key-value pair: $O(log(N))$
        \item Checking if the map contains a key: $O(log(N))$
        \item Checking if the map contains a value: $O(N)$
        \item Getting the value for a certain key; $O(1)$
        \item Finding the smallest or largest key: $O(log(N))$
        \item Finding the smallest key bigger than a given key: $O(log(N))$
        \item Finding the largest key smaller than a given key: $O(log(N))$
        \item Looping through the map: $O(N)$
    \end{itemize}

\section{Multisets and Multimaps}
    If you code in C++, there are versions of set and map that allow you to repeat values(Multiset and Multimap respectively). In java, there is no such data structure, but we can emulate a multiset using a map where the key stores the value and the value the key maps to store the number of times it occurs.

\section{Practice Problems}
    \begin{itemize}
        \item USACO 2015 January Contest, Silver: Stampede
        \item USACO 2015 December Contest, Platinum: High Card Low Card
    \end{itemize}

\end{document}