\documentclass[11pt]{article}
\usepackage{amsmath}
\usepackage{amssymb}
\usepackage{amsthm}
\usepackage{array}
\usepackage{color}
\usepackage{hyperref}
\usepackage{xy}
\usepackage[margin=0.9in]{geometry}
\usepackage{verbatim}
\parindent 0in

\usepackage{tikz}
\title{Welcome to Senior Computer Team!}
\author{SCT Officers}
\date{September 2021}

\begin{document}

\maketitle

\section{Introduction}
    Welcome to Senior Computer Team! We meet every Friday A block to learn algorithms and data structures and improve our problem solving skills through competitive programming. Our focus is on the USA Computing Olympiad (USACO) and ICPC-style competitions, but we also attend many other programming events. We look forward to having a great, fun year as a team! 
    
    
\begin{itemize}
  \item Sponsor: Dr. Yilmaz
  \item Captain: Andrew Wang
  \item Co-Captains: Pranav Mathur, Joshua Zhang
  \item Statistician: Aarav Bajaj
\end{itemize}

\section{Schedule}
    SCT will meet every Friday A block in Dr. Yilmaz's room (Room 202) and in Curie Commons (also consider going to ICT, which meets B block). Most blocks will follow the format of a 20-30 minute lecture followed by problem walkthroughs and strategies. Additionally, at least once a month, we will host in house contests, either during 8th period or over the weekend, for you to practice the skills you learn. Our complete schedule can be found at \url{https://activities.tjhsst.edu/sct/schedule}.
    
\section{Contests}
\subsection{In-House Contests}
       In-House Contests will be one of the most important ways to learn in SCT. We plan to have many In-House Contests this year in multiple formats to help you discover the weaknesses and strengths of your algorithmic programming skills. There will two types of contests: A live blitz round, which will be held during 8th period and focused on speed; and a longer comprehensive round, which will be open all weekend and focus on harder problems. We will alternate between these two formats to give practice with speed and problem solving. Later in the year, club members will have the opportunity to help write problems for contests.
\subsection{USACO}
    USACO, the most popular and prestigious computing contest in the US, is administered online through four contests in December, January, February, and ending with an US Open in March. There are four divisions of increasing difficulty: Bronze, Silver, Gold, and Platinum. Everyone starts out in the Bronze Division and has the opportunity to promote to higher divisions by scoring well in contests. Top competitors in the Platinum division are selected for USACO camp. We highly recommend everyone to take USACO contests as much of our material is centered aroiund it, and you can get a lot out of the contests.
\subsection{ICPC-style, Travel, and Codeforces Contests}
    The other main type of contest we prepare team members for is ICPC-style contests. We travel to several college competitions that use this format, such as PClassic (at UPenn), CMIMC (at Carnegie Mellon), UVA, and VTech. Our lectures and contests also teach topics and strategies that are useful in Codeforces contests, which you can learn more about at \url{https://codeforces.com/}.
    
\section{Lectures \& Guest Lecturing}
    Our goal at SCT is to teach you the competitive programming skills you need to be successful at competitions and even programming interviews later in your life. To that end, We work hard to deliver high-quality, engaging lectures for the club. Each of our lectures during the year will consist of 20-30 minutes of new content and the rest of the time will be spent showing examples and occasionally giving live code demos.\newline\newline
    Later in the year, we open up lecture writing to club members to explore advanced, and often non-conventional, topics in competitive programming (and computer science in general). Writing and delivering at least one guest lecture is a requirement to run for officer at the end of the year - and it's really fun! A new policy we're implementing this year is that \textbf{all guest lectures be submitted to the officers a week in advance for review}. This to help you improve your lecture writing and create better content for the club.
    
\section{TJIOI}
    Each spring, SCT hosts the TJ Intermediate Open in Informatics, a programming contest for high-schoolers. The contest features a theoretical written round in which competitors analyze complexity and develop algorithms conceptually, as well as a programming round - which is in  ICPC-style. This year will be the the first in several years that we are hosting TJIOI in-person, so we need your help! We need club members to help write problems and run the contest on the day of the event.

\section{Resources}
\subsection{Contact Us}
    Make sure to sign up for our mailing list and join our Facebook group! We will send out weekly announcements and updates. If you have any questions, feel free to contact us.
\begin{itemize}
  \item Website: \url{https://activities.tjhsst.edu/sct/}
  \item Email: tjhsstsctcaptains@gmail.com
  \item Facebook Group: \url{https://www.facebook.com/groups/tjsct/}
\end{itemize}

\subsection{Problem Archives}
    Solving problems is \textbf{the most important} part to learning new concepts. We will try to showcase a few problems relevant to each lecture topic at the end of lectures, but it is important that you do more problems!
\begin{itemize}
  \item USACO Training Pages
  \item USACO Past Contests Archives (Effectively learn through explanations)
  \item Codeforces
  \item Online Judges (UVA, Kattis, CodeChef)
\end{itemize}

\subsection{Competitive Programming Resources}
\begin{itemize}
  \item usaco.org
  \item Wikipedia
  \item geeksforgeeks.org
  \item \textit{Competitive Programming} by Steve and Felix Hamlin 
  \item \textit{Algorithms} by Sedgewick and Wayne
\end{itemize}
\end{document}
