\documentclass{article}
\usepackage[utf8]{inputenc}
\usepackage[margin=0.8in]{geometry}
\usepackage{hyperref}

\parindent 0in
\setlength{\skip\footins}{0.5cm}


\title{Welcome to Senior Computer Team!}
\author{SCT Officers}
\date{September 15, 2017}

\begin{document}

\maketitle

\section{Introduction}

Welcome to the TJHSST Senior Computer Team, widely regarded as one of the most awesome high school computer teams in the United States! We meet weekly to practice and enhance problem-solving skills involving programming. Our focus is on the USA Computing Olympiad (USACO), but we also compete in several other programming contests. Over the year, we hope that you will all remain with us as we learn
and compete together.

\begin{itemize}
	\item Sponsor: Ms. Galanos
	\item Captain: Justin Zhang
	\item Co-Captains: Mihir Patel, Daniel Wisdom
	\item Statistician: Srinidhi Krishnamurthy
\end{itemize}

We or any other returning members would be more than happy to answer any questions you may have about SCT, computer science, school, or life in general.


\section{First day signups}

First things first: please make sure you're signed up for both of the following.

\begin{enumerate}
    \item \textit{Mailing list.} We will use this to send out weekly announcements and lectures, so please read these emails! They are short and don't bite, we promise.
    \item \textit{Codeforces group.} This is where we will hold SCT contests. Codeforces also holds regular contests of its own, which we encourage you to do! If you do not have a Codeforces account, you should create one!
    \item \textit{YouTube channel.} This year SCT will be putting together a YouTube channel with all of the lectures recorded and edited. 
\end{enumerate}

If you aren't signed up for those things, please use the following link to sign up for today only.

\begin{center}
\Large{\url{https://goo.gl/m5ajcq}}
\end{center}


\section{Schedule}

SCT meets every Friday during A block. (ICT meets during B block, which you should consider joining as well.) A typical block consists of lectures focused on USACO topics. As has been done in past years, we will split lectures into two groups: one for bronze/silver and one for gold/platinum. We also will occasionally provide time for coding practice.

Our preliminary first semester lecture schedule is shown below.

\begin{center}
\begin{tabular}{ |c|c|c| }
\hline
Date & Bronze/Silver & Gold/Platinum\\
\hline
9/29 &	Computational Complexity & Graph Theory\\
10/6 &	Competitive Programming/Contest Formats	&Computational Geometry\\
10/20&	Recursion/DFS	&Union Find/Minimum Spanning Trees\\
10/27&	Prefix Sums	&BIT/Segment Trees\\
11/10&	DP (beginners)	&DP (advanced)\\
11/17&	DP (advanced)	&DP optimizations\\
12/1&	Sets and Maps/Hashing	&Line Sweep\\
12/8&	Shortest Paths	&Lowest Common Ancestor\\
12/15&	Binary Search	&Case Studies/Applications\\
\hline
\end{tabular}
\end{center}

\section{Contests}

\subsection{In-house contests}

We will hold interactive ``Case Study" sessions in SCT meetings before major contests (e.g. USACO), in which we will choose a couple of questions to academic solve (no coding). These will emphasize topics we think will be important on the contest, and will also serve as review for prior lectures.

We will also occasionally hold contests in our Codeforces group. Most will be practice contests, which are for your benefit. We may sometimes hold contests after school as well, depending on demand.

\subsection{USACO}

USACO is a competition administered online with monthly contests in December, January, and February, culminating in the US Open in April. There are four divisions in USACO: bronze, silver, gold, and platinum, each with their own set of problems for each month. All contestants start in the bronze division and advance by performing well in contests; top scorers are selected for USACO training camp. Our curriculum is centered around USACO, so we strongly encourage you to participate in it.


\subsection{Travel contests}

We will select 12 people to represent SCT at the following three contests (four people per contest), because team registration limits prevent us from sending more. Selections will be determined through competitive programming proficiency only. This may include USACO rankings and Codeforces ratings.

Each of the contests takes place on a Saturday in the spring (March or April), and is located on the corresponding university campus. They all have around 9-10 problems, and use ACM-style scoring; that is to say, the winner is determined by the total number of problems solved, and ties are broken by time penalty.

\begin{enumerate}
    \item \textit{VCU HSPC} --
    Both SCT and ICT attend. The contest length is 3 hours, and only Java is allowed.
    Website: \url{https://egr.vcu.edu/departments/computer/about/high-school-programming-contest/}

    \item \textit{UMD HSPC} --
    Only SCT attends this contest. The contest length is 3 hours, and only Java is allowed.
    Website:
    \url{http://www.cs.umd.edu/Outreach/hsContest.shtml}
    
    \item \textit{UVA HSPC} --
    FCT, ICT, and SCT all attend. The contest length is 4 hours, and Java, C, or C++ are permitted.
    Website: \url{http://acm.cs.virginia.edu/hspc.php}
\end{enumerate}


\subsection{Other university contests}

The following contests, however, do not have a team registration limit; you may therefore form teams on your own. The number of problems and scoring system is similar.

\begin{enumerate}
    \item \textit{PClassic} -- This contest takes place every fall (November) and spring (April) at the University of Pennsylvania. You may form teams of up to four, in either the novice or standard divisions. The contest length is 4 hours, and the programming language is Java or Python. Website: \url{http://pclassic.org/}.
    \item \textit{VT HSPC} -- This contest takes place in early December, on TJ campus through the Internet. You may form teams of up to \textbf{three}. The contest length is 4 hours, and the programming language is Java, Python, C, or C++. Website: \url{https://icpc.cs.vt.edu/}
\end{enumerate}


\section{TJIOI}

This year, we're once again hosting TJIOI, a programming contest for local high school students. We could definitely use your help in writing problems, sorting out logistics, designing t-shirts, etc. Please contact us at \verb|tjioiofficers@gmail.com| if you're interested! 


\section{First day contest}

For the first day, we've found some practice problems for you to do on Codeforces. The practice contest will be open through next Wednesday. You can find it by navigating to the Codeforces group, or to this URL:

\begin{center}
\Large{\url{http://codeforces.com/group/M4wsRWBHyZ/contest/215741}}
\end{center}


\section{Resources}

As mentioned earlier, make sure you're signed up for our mailing list and Codeforces group! If you are not in the TJHSST SCT Facebook group, please join that, too. We post announcements there frequently. Also don't forget to subscribe to the YouTube channel in case you miss any meetings or you want to explore past lectures. 


\begin{itemize}
    \item Codeforces group: \url{http://codeforces.com/group/M4wsRWBHyZ/}
    \item Facebook group: \url{https://www.facebook.com/groups/tjsct/}
    \item YouTube channel:
    \url{https://goo.gl/9fzBRd}
\end{itemize}

On our website, you can find links to many resources, including the above links, past lectures (including this one), and other resources, including Samuel Hsiang's excellent Crash Course Coding Companion.

\begin{center}
\Large{\url{https://activities.tjhsst.edu/sct/}}
\end{center}

If you have questions, concerns, feedback, anything — please don't hesitate to talk to any of the officers, or shoot us an email at \url{tjhsstsctcaptains@gmail.com}. We look forward to seeing you throughout the coming year!

\end{document}